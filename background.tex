
%---------------------------------------%
% Packages arranged by : Tsz Timmy Chan %
%                 Date : Dec 12th, 2019 % 
%---------------------------------------%

\documentclass{LSRIarticle}
\usepackage{LSRIcommon}


\begin{document}
%\smalltitle

\subsection{Motivation}
\textit{Below is a very rough version of a literature review that we have and points we draw from each and a sketch of where they go in a write up}
%MOTIVATION GOES HERE
Women 1.5 times more likely to leave STEM \parencite{ellisWomenTimesMore2016}.
Student retention: \parencite{dewitzCollegeStudentRetention2009}

National study on calculus \parencite{bressoudInsightsRecommendationsMAA}

\subsection{Literature Review}

\subsubsection{Active Learning}
\textbf{Spaces}: \parencite{parkTransformationClassroomSpaces2014, brooksSpaceConsequencesImpact2012, cotnerItNotYou2013}

SEMINAL: "Active Learning of Mathematics is defined as teaching methods and classroom norms that promote:
students’ deep engagement in mathematical reasoning;
peer-to-peer interaction; 
instructors' interest in and use of student thinking; and
instructors' attention to equitable and inclusive practices.
Curriculum should focus on key mathematical ideas (sense making & procedural fluency).

Students propose questions, communicate reasoning, & share solutions in process.

Instructors promote student engagement & build on student thinking." \parencite{laursenPrizeInquiryApproaches2019a}


Benefits (meta-analysis) "The data reported here indicate that active learning increases examination performance by just under half a SD and that lecturing increases failure rates by 55\%. The heterogeneity analyses indicate that (i) these increases in achievement hold across all of the STEM disciplines and occur in all class sizes, course types, and course levels; and (ii) active learning is particularly beneficial in small classes and at increasing performance on concept inventories" \parencite{freemanActiveLearningIncreases2014}. \textbf{This paper doesn't define active learning...?}


\subsubsection{Relationship between Metacognition and Confidence}
Generalized metacognitive knowledge as it relates to confidence: students have overconfidence and anxiety, which together affect metacognitive ability $\rightarrow$ math avoidance \parencite{ericksonMetacognitionConfidenceComparing2015}.


\subsubsection{Self Efficacy}

Defining articles: \parencite{hendyMeasurementMathBeliefs2014}
Confidence/interest/performance: gender patterns \parencite{ganleyMathematicsConfidenceInterest2016}
%\newpage



%\printbibliography[heading=bibintoc]


%\newpage
%\printglossary[type=\acronymtype]
%\printglossary





\end{document}
